\documentclass{article}
\usepackage{graphicx}
\usepackage{algorithm}
\usepackage[noend]{algorithmic}
\usepackage{subfigure}
\usepackage{amssymb, amsmath, graphicx, charter, latexsym}
\usepackage{layouts}
\usepackage[letterpaper]{geometry}
\usepackage{enumerate}
\usepackage{epstopdf}
\usepackage{ragged2e}
%\usepackage{times}
\usepackage{mathtools}
%\usepackage[scaled]{helvet}
\usepackage{mathptmx}
\usepackage{verbatim}
\usepackage{listings}
\usepackage{siunitx}

\lstset{
basicstyle=\ttfamily,
}

\begin{document}

\title{\bf ECEN 689-- Real-Time Wireless Networks: Project 2\\ (Due on 4/1)}
\date{}
\author{%
Ping-Chun Hsieh\\
\texttt{lleyfede@tamu.edu}
\and
Tao Zhao\\
\texttt{alick@tamu.edu}
\and
Dongni Han\\
\texttt{handongni2015@tamu.edu}
}
\maketitle

\section*{Terminology}

In our report, we use ``server'' to denote the WiFi access point (AP), and
``client'' to denote the terminal device such as a mobile phone, a tablet, and
so on. Throughout our simulation, we let node $0$ be the server, and node $1$ be
the client, which correspond to the device A and B in the problem descriptions
respectively.

S-WiFi is the name of our application as well as our project. It stands for
Smart WiFi, or whatever you think it is.

\section*{Simulation Setup}
\begin{table}[htbp]
\centering
    \caption{Parameters of the wireless channel.}
    \vspace{2mm}
    \begin{tabular}{ | l | l | }
    \hline
    Item & Value \\ \hline
    Path loss exponent & 2.0  \\ \hline
    Shadowing deviation & \SI{4.0}{dB} \\ \hline
    Reference distance & \SI{1.0}{m} \\
    \hline
\end{tabular}
\label{table: channel}
\end{table}
Throughout the report, we consider a single wireless link between two devices, say A and B. The transmitter power level is \SI{10}{mW}. We use the shadowing module as the wireless channel. The parameters of the channel are summarized in Table \ref{table: channel}.

\begin{table}[htbp]
\centering
\caption{Parameters of the 802.11b MAC.}
    \vspace{2mm}
    \begin{tabular}{ | l | l | }
    \hline
    Item & Value \\ \hline
    Data rate & \SI{11}{Mb/s}  \\ \hline
    Basic rate & \SI{1}{Mb/s}  \\ \hline
    PLCP data rate & \SI{1}{Mb/s}  \\ \hline 
    Preamble length & \SI{144}{bits} \\ \hline
    Slot time & \SI{20}{\mu s} \\ \hline
    SIFS & \SI{10}{\mu s} \\
    \hline
\end{tabular}
\label{table: mac}
\end{table}

For the medium access control (MAC) layer, we use the 802.11 module built in ns-2. Following the IEEE 802.11b standard, the MAC layer parameters are chosen as in Table \ref{table: mac}.


\section*{Uplink Transmissions with PCF}
\label{section: uplink}



\section{Baseline Policy}
\label{section: baseline}


\frenchspacing In PCF mode, the AP decides which client can transmit: The AP will first send a POLL packet to the selected client. A client can only transmit its packet after it receives the POLL packet from the AP. This allows the AP to have full control over which client transmits. 

Baseline policy: At the beginning of each interval, the AP asks each client, one by one, the number of packets that it generates. After this process, the AP also knows the number of packets at each client, and it can make the best decision.

In our SWiFi project, the way we implement baseline policy is as follow. 
First we need judge whether the device is a server or a client.
If it is a server and the value of  \lstinline |do_pull_num|  is true,  \lstinline |poll_state_|  will be set  \lstinline |SWiFi_POLL_NUM| , which means that AP will send a POLL packet to ask each client one by one implemented by using function \lstinline |scheduleRoundRobin|. 
If it is a server but the value of  \lstinline |do_pull_num|  is false,  \lstinline |poll_state_|  will be set  \lstinline |SWiFi_POLL_DATA| , which means that AP will send data to a specified client scheduled by maxweight policy. 

If it is s client,  all clients generate a number of packets  at the beginning of each interval implemented in swifi.tcl file. By using maxweight policy, each client will calculate the queue and the server will choose the client with the largest queue. There is one special condition, there may be no client left in one interval. In this case, \lstinline |poll_state_| will be set \lstinline |SWiFi_POLL_IDLE|.

\section{Implementation in NS-2}
\label{section: ns2}


\section{Simulation Results}
\label{section: simulation}
\subsection{Symmetric System with 2 Clients}
Under the simulation setup described above, the real-time 
Real-time and non-real-time traffic
\subsection{Asymmetric System with 2 Clients}

\subsection{System with N Clients}
%\begin{figure}[htbp]
%\centering
%\subfigure[Downlink round-trip time.]{
%\includegraphics[scale = 0.7]{downlink_rtt.pdf}}
%\subfigure[Uplink round-trip time.]{
%\includegraphics[scale = 0.7]{uplink_rtt.pdf}}
%\caption{Round-trip time when node A and node B are close to each other.}
%\label{figure: rtt}
%\end{figure}


\end{document}